\documentclass[]{article}
\usepackage[T1]{fontenc}
\usepackage[italian]{babel}
\usepackage{hyperref}
\usepackage{amsmath}

\begin{document}
\title{Simulazione di stormi}
\author{Francesco Giacomini}
\maketitle

\tableofcontents

\section{Descrizione del problema}

Il progetto consiste nell'implementazione di una simulazione del comportamento degli stormi di uccelli in volo in uno
spazio bidimensionale.

Reso noto da un software di intelligenza artificiale realizzato nel 1986, la simulazione si basa sull'interazione tra
agenti detti \emph{boids} in uno spazio. Nel modello più elementare gli agenti determinano la loro traiettoria tramite tre
regole:
\begin{enumerate}
\item \emph{separazione}: il boid si allontana dai boids vicini
\item \emph{allineamento}: il boid tende ad allinearsi alle traiettorie dei boids vicini
\item \emph{coesione}: il boid tende a muoversi verso il baricentro dei boids vicini
\end{enumerate}
Modelli più complessi introducono altre regole, per esempio per evitare gli ostacoli o per gestire stormi differenti.

\section{Controllo dei vicini}

Le regole di volo dei boids vengono applicate tenendo conto dei boidi "vicini".
La vicinanza tra boids può essere determinata in base alla loro posizione. Dato un boid $b_i$ i suoi vicini sono tutti i boids $b_j$ per cui:

$$|\vec{x}_{b_i}-\vec{x}_{b_j}|<d$$

\section{Le regole di volo}

Le regole di volo dei boids vengono usate per determinare ognuna una componente della variazione di velocità del boid.

Per ogni boid $b_i$:

$$\vec{v}_{b_i}=\vec{v}_{b_i}+\vec{v}_1+\vec{v}_2+\vec{v}_3$$

$$\vec{x}_{b_i}=\vec{x}_{b_i}+\vec{v}_{b_i} \Delta t$$

dove $\vec{v}_1$, $\vec{v}_2$ e $\vec{v}_3$ sono le velocità ottenute dall'applicazione delle 3 regole.

\subsection{Regola 1: separazione}

La regola ha lo scopo di evitare che i boids collidano tra di loro.

$$\vec{v}_1 = -s \sum_{j\ne i}(\vec{x}_{b_i}-\vec{x}_{b_j}) \text{\quad se\quad } |\vec{x}_{b_i}-\vec{x}_{b_j}|<d_s$$

Usando la distanza come fattore per determinare la nuova velocità permette di avere una accelerazione graduale del boid che si allontana. Un opportuno fattore di separazione $s$ determina l'intensità della repulsione. La distanza $d_s$ stabilisce il range di influenza della regola.

\subsection{Regola 2: allineamento}

Lo scopo della regola è fare in modo che i boids procedano nella stessa direzione dello stormo.

$$\vec{v}_2 = a(\frac{1}{n-1}\sum_{j\ne i}\vec{v}_{b_j} - \vec{v}_{b_i})$$

Viene sottratta alla media delle velocità dei boids (escluso il boid su cui è applicata la regola) la velocità attuale del boid $b_i$. Il tutto moltiplicato per un fattore di allineamento $a<1$ che determina la rapidità con cui il boid sterza.

\subsection{Regola 3: coesione}

La regola induce il boid a volare verso il centro di massa dei boid vicini. Il centro di massa è dato da:

$$\vec{x}_{c} = \frac{1}{n-1}\sum_{j\ne i}\vec{x}_{b_j}$$

La velocità $\vec{v}_3$ si ottiene sottraendo la posizione attuale del boid alla posizione del centro di massa, moltiplicando poi il risultato per un fattore di coesione $c$ opportuno:

$$\vec{v}_3 = c(\vec{x}_{c}-\vec{x}_{b_i})$$

\subsection{Comportamento ai bordi}

Si suggerisce di implementare una regola per determinare il comportamento di un boid
quando raggiunge un bordo dell'area di simulazione. Due esempi di strategia di implementazione
posso essere:

- Lo spazio bidimensionale è toroidale, imponendo la periodicità dell'area di simulazione: se un boid supera un bordo
dell'area di simulazione, viene "teletrasportato" al lato opposto.

- Si può introdurre una regola di volo che spinge un boid ad allontanarsi dai bordi: se un boid supera un bordo dell'area di simulazione, gli viene sommata una velocità diretta verso il bordo opposto.

\section{Implementazione del modello}

Il programma accetta in input almeno il numero di boids e i parametri $s$, $a$, e $c$ delle regole di volo.

L'output del programma dovrebbe mostrare l'andamento nel tempo della distanza media tra boids e la sua deviazione standard, e l'andamento nel tempo della velocità media dei boids e la sua deviazione standard.

Se propriamente simulati, i boids nel tempo dovrebbero tendere ad avvicinarsi e ad avere velocità mediamente simili tra di loro, assumendo il comportamento di uno stormo coeso.

\section{Variazioni sul tema}

È possibile introdurre ulteriori elementi per rendere
più elaborato il modello. A titolo di esempio:

- Mostrare l'andamento nel tempo della distribuzione dei boids nel piano con degli istogrammi.

- Cambiare il modo in cui vengono individuati i vicini fornendo ai boids un angolo di vista. \href{esempio}{http://www.red3d.com/cwr/boids/images/neighborhood.gif}

- Utilizzare una libreria grafica per mostrare il volo dei boids nel piano

- Simulare il comportamento di stormi differenti, limitando le regole di volo ai boids appartenenti allo stesso stormo

- Simulare il comportamento di un boid predatore che tende a seguire boids appartenenti ad altri stormi, i quali applicheranno su di esso la sola regola di separazione

\section{Suggerimenti sull'implementazione}

I parametri $s$, $a$, e $c$ delle regole di volo dipendono dall'implementazione delle velocità. In linea di principio
sono dei fattori di proporzionalità che determinano l'influenza di una certa regola sul moto dei boids. Un suggerimento
è provare in partenza con valori compresi tra 0 e 1, e trovare il range che più si adatta alla propria implementazione.

Il parametro $d$ determina la vicinanza tra boids e di conseguenza l'applicazione delle regole di volo; si consiglia di
scegliere un valore che permetta ai boids di "vedere" i vicini più prossimi. Valori troppo alti risulteranno in voli più
caotici, mentre valori troppo bassi renderebbero i boids "ciechi". In entrambi i casi sarà difficile il formarsi di
stormi.

Il parametro $d_s$ determina la soglia di separazione tra boids, il suo valore deve essere molto minore del valore $d$
di vicinanza. In tal modo i boids potranno tenersi vicini al proprio stormo, senza però sovrapporsi.

E' opportuno introdurre un limite alla velocità massima di volo dei boids per evitare comportamenti anomali che
potrebbero rendere difficile la simulazione. Essendo la simulazione a intervalli di tempo discreti (per forza di cose,
facendo la simulazione su una macchina digitale) dei valori di velocità troppo alti possono rendere impossibile per i
boids "incontrarsi" e quindi adattare il proprio volo ai vicini.

\section{Riferimenti utili}

- \href{Boids - wikipedia}{https://it.wikipedia.org/wiki/Boids}
- \href{Pagina di Craig Raynolds sui boids}{http://www.red3d.com/cwr/boids/}
\end{document}